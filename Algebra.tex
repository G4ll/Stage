\documentclass[a4paper]{article}

\makeatletter
\title{Algebra}\let\Title\@title
\author{Andrea Gallese}\let\Author\@author
\date{\today}\let\Date\@date

%\usepackage[italian]{babel}
\usepackage[utf8]{inputenc}

\usepackage{mathtools}
\usepackage{amssymb}
\usepackage{amsthm}
%\usepackage{faktor}
%\usepackage{wasysym}
%\usepackage{thmtools}

\usepackage[margin=1.5cm]{geometry}
\usepackage{fancyhdr}
\usepackage[position=top]{subfig}
\usepackage{multirow}

\usepackage{lipsum}
\usepackage{titlesec}
\usepackage{multicol}
\usepackage{setspace}
\usepackage{mdframed}
\usepackage{enumitem}

% Comando Intitolante
\newcommand{\Intitola}{\begin{center}
		\vspace*{0,5 cm}
		{\Huge \textsc{\Title}} \\
		\vspace{0,5 cm}
		\textsc{\Author} \hspace{1cm} \textsc{\Date}
		\thispagestyle{empty}
		\vspace{0,7 cm}
\end{center}}

% magia
\iffalse
\usepackage{hyperref}
\hypersetup{
	colorlinks,
	citecolor=black,
	filecolor=black,
	linkcolor=black,
	urlcolor=black
}
\fi

% Formato Teoremi, Dimostrazioni, Definizioni
\newtheorem{theorem}{Teorema}[section]
\newtheorem{lemma}[theorem]{Lemma}
\theoremstyle{remark}
\newtheorem*{remark}{Osservazione}
\theoremstyle{definition}
\newtheorem*{definition}{Definizione}
\renewcommand\qedsymbol{$\clubsuit$}

% Frontespizio e piè di pagina
\pagestyle{fancy}
\fancyhf{}
\rhead{\textsf{\Author}}
\chead{\textbf{\textsf{\Title}}}
\lhead{\textsf{\today}}

% Frontespizio e piè di pagina
\pagestyle{fancy}
\fancyhf{}
\rhead{\textsf{\Author}}
\chead{\textbf{\textsf{\Title}}}
\lhead{\textsf{\today}}
\cfoot{\thepage}

% Crea una nuova pagina per ogni sottosezione
\newcommand{\subsectionbreak}{\clearpage}

% Per avere le sezioni con le lettere
\renewcommand{\thesection}{\Alph{section}}

%Comandi specifici
\newcommand{\N}{\mathbb{N}}
\newcommand{\Z}{\mathbb{Z}}
\newcommand{\Q}{\mathbb{Q}}
\newcommand{\R}{\mathbb{R}}
\newcommand{\K}{\mathbb{K}}
\newcommand{\F}{\mathbb{F}}

\renewcommand{\S}{\mathcal{S}}

\newcommand{\Aut}[1]{\mathrm{Aut}\left( #1 \right)}
\newcommand{\Int}[1]{\mathrm{Int}\left( #1 \right)}
\newcommand{\Orb}[1]{\mathcal{O}rb\left( #1 \right)}
\newcommand{\Stab}[1]{\mathcal{S}tab\left( #1 \right)}
\newcommand{\gen}[1]{\langle #1 \rangle}
\newcommand{\Gal}[1]{\mathcal{G}al\left( #1 \right)}

\newcommand{\LR}{\quad\Leftrightarrow\quad}
\newcommand{\RR}{\quad\Rightarrow\quad}

\newcommand{\fun}[5]{
	\begin{align*}
	#1 \colon #2 &\to #3 \\
	#4 &\mapsto #5
	\end{align*}
}

% indentazione
\setlength{\parindent}{0pt}

% multicols
\usepackage{multicol}
\setlength\columnsep{20pt}
\setlength{\columnseprule}{0,5pt}

% Per disegnare diagrammi commuatativi
\usepackage{tikz-cd}
\usepackage{tikz}


% Bullet delle liste puntate
\renewcommand\labelitemi{$ \blacktriangleright $}

\begin{document}
\Intitola
\small
\begin{multicols}{2}
	\begin{center}
		\textsc{Cos'è un polinomio}
	\end{center}

	Presentiamo il nostro amico polinomio
	\[ P(x) = a_nx^n + a_{n-1}x^{n-1} + \dots + a_1 x + a_0 \]
	buttiamoci un sacco di esempi
	\begin{equation*}
	\begin{array}{c}
	x^2 - 5x +6\\
	3x^3 - 6x^2 -3x + 6\\
	4x^4 +1\\
	x^6 +2x^5 + 2x^4 +2x^3 + 2x^3 + 2x^2 + 2x +1
	\end{array}
	\end{equation*}
	
	e fattorizziamoli
	
	\begin{equation*}
	\begin{array}{c}
	(x-2)(x-3) \\
	3(x-1)^2(x+1) \\
	(2x^2 + 2x +1)(2x^2 -2x +1) \\
	(x+1)^2(x^2+x+1)(x^2-x+1)
	\end{array}
	\end{equation*}
	
	\textcolor{gray}{A ogni polinomio è associata in modo naturale una funzione polinomiale: la valutazione.} \\
	
	\textcolor{gray}{\textbf{Ruffini.} Se $ P(\alpha) = 0 $, allora $ P(x) = (x-\alpha)Q(x) $.} \\
	
	\textcolor{gray}{\textbf{Problema 1.} Quante sono le coppie di numeri reali $ (x,y) $ che soddisfano entrambe le equazioni $ x + y^2 = y^3 $ e $ y + x^2 = x^3 $? \\}
	
	\textbf{Teorema Fondamentale dell'Algebra.} Se permettiamo $ \lambda_i \in \mathbb{C} $, allora ci è concesso scrivere
	\[ P(x) = a_n(x-\lambda_1)(x-\lambda_2) \cdots (x-\lambda_n) \]
	
	\textcolor{gray}{\textbf{Problema 2.} Si sa che $ p(x) $ è un polinomio monico di grado 5. Inoltre, si sa che le soluzioni dell’equazione $ p(x) = 0 $ sono esattamente $ x = 0, 1, 2, 4 $. Determinare il massimo valore che può assumere il coefficiente del termine di primo grado. [Feb 12]}\\
	
	\textbf{Problema 5.} Sapendo che il polinomio $ p $ è tale che, per ogni intero $ n $, $ p(5^n - 1) = 5^{5n} - 1 $, quanto varrà $ p(3) $? [Feb 13?] \\
	
	\textbf{Principio di identità.} Se due polinomi $ P(x) $ e $ Q(x) $ assumono lo stesso valore in più punti del massimo dei loro gradi, allora coincidono. \\
	
	\textbf{Problema 6.} Abbiamo un polinomio monico $ P $ di grado $ 99 $, trovare $ P(101) $ sapendo che
	\[ P(k) = \frac{1}{k} \quad \text{per } k = 1, 2, \dots, 100 \]
	
	\textbf{Formule di Viète.} \\
	
	\textbf{Problema 3.} Abbiamo
	\[ P(x) = x^{2017} + 2x^{2016} +3x^{2015} + \dots + 2016x + 2017 \]
	Trovare la somma dei quadrati delle radici, la somma degli inversi delle radici \textcolor{gray}{e anche la somma degli inversi delle radici aumentate di 1.} \\
	
	\textbf{Problema 4.} Calcolare la somma dei coefficienti del polinomio
	\[ (x^{21} + 4x^2 - 3)^{2001} - (x^{21} + 4x^2 - 3)^{667} +x^{21} +4x^2. \]
	
	\textcolor{gray}{\textbf{Problema 4.} Il polinomio $ P(x) $, di grado 42, assume il valore 0 nei primi 21 numeri primi dispari e nei loro reciproci. Quanto vale il rapporto $ P(2)/P(1/2) $?}\\
	
	
	\textcolor{gray}{\textbf{Problema 7.} Siano $ a, b, c $ interi nonnulli tali che sia $ \frac{a}{b} +\frac{b}{c}+ \frac{c}{a} $ che $ \frac{a}{c} +\frac{c}{b}+ \frac{b}{a} $ sono interi. Dimostra che $ |a| = |b| = |c| $.} \\
	
	\textcolor{gray}{\textbf{Problema 8.} Per il suo gioco preferito Andrea ha bisogno di usuali dadi a sei facce. Purtroppo Andrea ha perso i dadi originali, ma possiede un dado a 4 facce e un dado a 9 facce (la forma del dado non è rilevante, si sa che ogni faccia ha la stessa probabilità di uscire). In quali modi Andrea può disegnare i puntini sulle facce dei due dadi (mettendo almeno un puntino per faccia) in modo che lanciandoli e sommando i puntini si ottengano gli stessi esiti, con le stesse frequenze, che si otterrebbero con due ordinari dadi da 6?}
	
\end{multicols}

\begin{center}
	\textsc{L'anello dei polinomi a coefficienti interi}
\end{center}

\begin{multicols}{2}
	\textcolor{gray}{\textbf{Problema 1.} (Gas 2012) Le carte della $ \mathbb{Q} $ di cuori sono molto preoccupate: non passa giorno che non si oda il grido ”Tagliategli la testa!”. Tengono anche un resoconto
	di quante esecuzioni capitali ci sono ogni giorno. Ultimamente hanno registrato questi dati:
	$ d(2002) = 11, \, 
	d(2006) = 7, \, 
	d(2008) = 5,\, 
	d(2009) = 4,\,
	d(2011) = 2 ,\, 
	 $ dove $ d(t) = h $ è la funzione decapitazione e indica che sono state tagliate h teste nel giorno t del resoconto (un numero negativo indica persone
	graziate). Ultimamente le carte si sono rese conto che per uno schiribizzo della $ \mathbb{Q} $ di cuori, d è un polinomio a coefficienti interi. Oggi è il
	giorno 2013 del resoconto e Q è molto nervosa per la partita di croquet, per cui il numero sarà inevitabilmente un intero positivo. Quante teste
	rotoleranno oggi come minimo?} \\
	
	
	\textbf{Problema 1.} (Gas 2017) Uno dei matematici rivelò agli altri che stava studiando i polinomi, nella speranza di trovare una formula per il terzo grado.
	“Durante il mio lavoro, mi sono imbattuto in certi polinomi molto particolari: se chiamiamo $ n $ la somma dei coefficienti di
	$ p(x) $, allora $ p(2017) = n! $. Qual è il più grande $ n  $ minore di 10000 tale per cui esista un polinomio $ p $ a coefficienti interi che
	soddisfi queste condizioni?”. \\
	
	\textbf{Problema 1.} Siano $ a,\, b,\, c $ tre interi distinti, e sia $ P $ un polinomio a \emph{coefficienti interi}. Mostrare che le condizioni $$  P(a) = b,\quad P(b) = c,\quad P(c) = a,  $$ non possono essere soddisfatte simultaneamente. \\
	
	\textbf{Problema 1.} Let $ f (x) $ be a monic polynomial with integral coefficients. If there are four different
	integers $ a, b, c, d $, so that $ f (a) = f (b) = f (c) = f (d) = 5 $, then there is no
	integer $ k $, so that $ f (k) = 8 $. \\
	
	Soluzioni razionali di polinomi a coefficienti interi? \\
	
	\textbf{Problema 3.} Quante sono le coppie di numeri reali $ (x,y) $ che soddisfano entrambe le equazioni $ x + y^2 = y^3 $ e $ y + x^2 = x^3 $?
	
	\begin{center}
		\textsc{Filtri dell'unità}
	\end{center}
	
	\textbf{Problema 4.} Calcolare la somma dei coefficienti del polinomio
	\[ (x^{21} + 4x^2 - 3)^{2001} - (x^{21} + 4x^2 - 3)^{667} +x^{21} +4x^2. \]
	
	\textbf{Problema 6.} Il Bossel finale del primo livello sarà un polinomio $ f (x) $ scelto dal malvagio Brouwer tra quelli a coefficienti interi
	della forma $ a_5x^5 + a_4 x^4 + a_3 x^3 + a_2 x^2 + a_1 x + a_0 $ tali che $ a_0 - 3a_2 + 9a_4 = 0 $ e $ a_1 - 3a_3 + 9a_5 = 0 $, e per sconfiggerlo
	Mathio dovrà fattorizzare il numero $ f (33) $. Taod, il fungo antrolomorfo, vuole aiutare Mathio: per ogni primo $ p $ che
	sicuramente dividerà $ f (33) $ lascia in un punto del livello un foglio con scritto il numero p. Mathio raccoglie i fogli e
	per esercizio calcola la somma dei numeri scritti. Sapreste farlo anche voi? \\
	
	\textbf{Problema 9.} A sequence $ a_1, a_2, \dots, a_n $ is called $ k $-balanced if $ a_1 + a_{k+1} + \dots = a_2 + \dots + a_{k+2} + \dots = \dots = 
	a_k +a_{2k} + \dots $. Suppose the sequence $ a_1, a_2, \dots , a_{50} $ is $ k $-balanced for $ k = 3,\, 5,\, 7,\, 11,\, 13,\, 17 $. Prove
	that all the values $ a_i $ are zero. \\
	
	\begin{center}
		\textsc{Polinomi per altri scopi}
	\end{center}
	
	Presentare il dado! \\
	
	\textbf{Problema 7.} Siano $ a, b, c $ interi nonnulli tali che sia $ \frac{a}{b} +\frac{b}{c}+ \frac{c}{a} $ che $ \frac{a}{c} +\frac{c}{b}+ \frac{b}{a} $ sono interi. Dimostra che $ |a| = |b| = |c| $. \\
	
	\textbf{Problema 8.} Per il suo gioco preferito Andrea ha bisogno di usuali dadi a sei facce. Purtroppo Andrea ha perso i dadi originali, ma possiede un dado a 4 facce e un dado a 9 facce (la forma del dado non è rilevante, si sa che ogni faccia ha la stessa probabilità di uscire). In quali modi Andrea può disegnare i puntini sulle facce dei due dadi (mettendo almeno un puntino per faccia) in modo che lanciandoli e sommando i puntini si ottengano gli stessi esiti, con le stesse frequenze, che si otterrebbero con due ordinari dadi da 6? \\
	
	\textbf{Problema 5.} An unfair coin has a 2/3 probability of turning up heads. If this coin is tossed 50 times, what is the probability that the total number of heads is even? \\

	
\end{multicols}

\begin{center}
	\textsc{Disuguaglianze}
\end{center}

\begin{multicols}{2}
	\textbf{Teorema 0.} I quadrati sono nonnegativi: $ x^2 \geq 0. $ \\
	
	\textbf{Problema 5.} Let $ c $ be the length of the hypotenuse of a right angle triangle whose other two sides have
	lengths $ a $ and $ b $. Prove that $ a + b \geq 2c $. When does equality hold? (Canada, 1969)\\
	
	\textbf{Problema 4.} Dimostrare che $ (1+x)^n \geq 1+nx $ per ogni $ n \geq 1 $ intero e $ x > -1 $ reale.\\
	
	\textbf{Problema 1.} Dimostrare che per ogni scelta di $ a, \, b , \, c $ reali vale
	\[ a^2 + b^2 + c^2 \geq ab + bc + ca. \]
	
	\textbf{Teorema 1.} (AM-GM) Per ogni $ n $-upla di numeri reali positivi vale
		\[ \sqrt[n]{a_1a_2\dots a_n} = \frac{a_1 + a_2 + \dots + a_n}{n}. \]
	
	\textbf{Problema 2.} Presi tre reali positivi $a$, $b$, $c$ tali che $a+b+c = 36$, quanto può valere al massimo $abc$? Quanto può valere al massimo $a^3b^2c$? \\
	
	\textbf{Problema 6.} Trova, tra tutti i triangoli di perimetro $2$, quello di area massima.\\
	
	\textbf{Problema 3.} The diagonals of a convex quadrilateral intersect in $ O $. What is the smallest area
	this quadrilateral can have, if the triangles $ AOB $ and $ COD $ have areas 4 and 9,
	respectively?\\
	
	\textbf{Problema 7.} Show that $ (n + 1)^n \geq 2^n \cdot n! $ for $ n = 1,\, 2,\, 3,\, \dots $ \\
	
	\textbf{Problema 8.} Sia $ n $ un intero positivo. Un treno ferma in $ 2n $ stazioni, incluse quella iniziale e finale, numerate in
	ordine dalla prima alla $ 2n $-esima. Si sa che in una certa carrozza, per ogni coppia di interi $ i,\, j $ tali
	che $ 1 \leq i < j \leq 2n $, è stato prenotato esattamente un posto per il tragitto tra la stazione $ i $-esima
	e quella $ j $-esima. Ovviamente prenotazioni diverse non possono sovrapporsi. Determinare, in
	funzione di $ n $, il numero minimo di posti che devono essere disponibili in quella carrozza affinché
	la situazione descritta sia possibile.
	
	
	
\end{multicols}

\textbf{Problemi.}

\begin{enumerate}
	\item Due polinomi monici (cioè con coefficiente di grado massimo uguale a 1) a coefficienti interi $ p(x) $ e $ q(x) $ sono tali che il loro massimo comun divisore sia $ (x - 1)(x - 2) $, il loro minimo comune multiplo sia $ (x-1)^2 (x-2)^3(x-3)(x+ 1)$ e il grado di $ p(x) $ sia minore o uguale al grado di $ q(x) $. In quanti modi può essere scelto $ p(x) $?\\
	(A) 4 (B) 5 (C) 8 (D) 10 (E) 12 [febboh]
	
	\item Il polinomio $ P(x) $, di grado 42, assume il valore 0 nei primi 21 numeri primi dispari e nei loro reciproci. Quanto vale il rapporto $ P(2)/P(1/2) $?\\
	(A) 0 (B) 1 (C) $ 2^{21} $ (D) $ 3^{21} $ (E) $ 4^{21} $ [feb17]
	
	\item Quante sono le coppie di numeri reali $ (x,y) $ che soddisfano entrambe le equazioni $ x + y^2 = y^3 $ e $ y + x^2 = x^3 $?\\
	(A) 1 (B) 3 (C) 5 (D) 9 (E) Infinite [Feb17]
	
	\item Sapendo che il polinomio $ p $ è tale che, per ogni intero $ n $, $ p(5^n - 1) = 5^{5n} - 1 $, quanto varrà $ p(3) $? [Feb 13?]
	
	\item Due numeri $ a $ e $ b $ sono tali che $ \dfrac{3a+b}{a-b} = 2 $, quanto vale $ \dfrac{a^3}{b^3} $? [Feb 12]
	
	\item Si sa che $ p(x) $ è un polinomio monico di grado 5. Inoltre, si sa che le soluzioni dell’equazione $ p(x) = 0 $ sono esattamente $ x = 0, 1, 2, 4 $. Determinare il massimo valore che può assumere il coefficiente del termine di primo grado. [Feb 12]
	
	\item Siano $ p(x) $ e $ q(x) $ due polinomi distinti di grado minore o uguale a 3, a coefficienti interi e tali che$$  
	p(1) = q(1), p(2) = q(2), p(3) = q(3),
	p(-1) = -q(-1), p(-2) = -q(-2), p(-3) = -q(-3).
	 $$	Qual è il minimo valore che può assumere $ [p(0)]^2 + [q(0)]^2 $ ? [Feb12]
	 
	 \item  (IMO 2006) Let $ P(x) $ be a polynomial of degree $ n > 1 $ with integer coefficients and let $ k $ be a positive integer. Consider the polynomial
	 $$  Q(x) = \underbrace{P(P(\dots (P(x))\dots)}_{k \text{ times }}.  $$
	 Prove that there are at most $ n $ integers $ t $ such that $ Q(t) = t $.
	 
	 \item Find a point $ P $ inside the triangle $ ABC $, such that the product $ P L \cdot P M \cdot P N $ is
	 maximal. Here $ L $, $ M $, $ N $ are the feet of the perpendiculars from $ P $ onto $ BC $, $ CA $,
	 $ AB $ (BrMO 1978).
	 
	 \item Fifty watches, all showing correct time, are on a table. Prove that at a certain moment
	 the sum of the distances from the center $ O $ of the table to the endpoints of the minute
	 hands is greater than the sums of the distances from $ O $ to the centers of the watches
	 (AUO 1976).
	 
	 \item The diagonals of a convex quadrilateral intersect in $ O $. What is the smallest area
	 this quadrilateral can have, if the triangles $ AOB $ and $ COD $ have areas 4 and 9,
	 respectively?
	 
	 \item The product of three positive reals is 1. Their sum is greater than the sum of their
	 reciprocals. Prove that exactly one of these numbers is $ > 1 $.
	 
	 \item Show that $ (n + 1)^n \geq 2^n \cdot n! $ for $ n = 1,\, 2,\, 3,\, \dots $ 
	 
	 \item Let $ m $ and $ n $ be positive integers. Find the minimum value of
	 \[ x^m + \frac{1}{x^n} \]
	 for $ x > 0 $.
	 
	 \item The Fibonacci sequence is $ F_n = 1,\, 1,\, 2,\, 3,\, \dots $. Prove that
	 \[ \frac{1}{2} + \frac{1}{2^2} + \frac{2}{2^3} + \frac{3}{2^4} + \frac{5}{2^5} + \dots + \frac{F_n}{2^n} \leq 2 \]
	 
	 \item Let $ a $, $ b $, $ c $ be the sides of a triangle, and $ T $ its area. Prove: $ a^2+b^2+c^2 \geq 4\sqrt{3}T $. In what case does
	 equality hold? (IMO, 1961)
\end{enumerate}

\newpage
\newgeometry{left=3cm,bottom=0.1cm}
\begin{center}
	\vspace*{0,5 cm}
	{\Huge \textsc{Problemi di Algebra}} \\
	\vspace{0,5 cm}
	\textsc{\Author} \hspace{1cm} \textsc{23 Febbraio 2017 - Pavia}
	\thispagestyle{empty}
	\vspace{0,7 cm}
\end{center}
\normalsize


1. Il polinomio $ x^3 -26x^2 + 203x - 521 $ ha radici $a, b, c$ calcola 
\[ a^2 + b^2 + c^2 \qquad\text{e}\qquad (a+b)(b+c)(c+a) \qquad\text{e}\qquad \frac{1}{a+1}+\frac{1}{b+1}+\frac{1}{c+1}\]

2. Presi tre reali positivi $a$, $b$, $c$ tali che $a+b+c = 36$, quanto può valere al massimo $abc$? Quanto può valere al massimo $a^3b^2c$? \\

3. Trova tutti i polinomi $ P(x) $ tali che
\[ P(P(x)) - x^2 = x\cdot P(x) \]

4. Mostrare che per ogni numero naturale $ n \geq 1 $ vale
\[ 1 \cdot 3 \cdot 5 \cdots (2n-1) \leq n^n \]

5. Abbiamo un polinomio $ P $ di grado $ 99 $, trovare $ P(101) $ sapendo che
\[ P(k) = \frac{1}{k} \quad \text{per } k = 1, 2, \dots, 100 \]

6. Trova, tra tutti i triangoli di perimetro $2$, quello di area massima. \\

\newpage
\begin{center}
	\vspace*{0,5 cm}
	{\Huge \textsc{Algebra}} \\
	\vspace{0,5 cm}
	\textsc{\Author} \hspace{1cm} \textsc{15 Febbraio 2018 - La Spezia}
	\thispagestyle{empty}
	\vspace{0,7 cm}
\end{center}
\normalsize

\begin{enumerate}
	\item  Il polinomio $ x^3 -26x^2 + 203x - 521 $ ha radici $a,\, b,\, c$ calcola 
	\[ a^2 + b^2 + c^2 \qquad\text{e}\qquad (a+b)(b+c)(c+a) \qquad\text{e}\qquad \frac{1}{a+1}+\frac{1}{b+1}+\frac{1}{c+1}\]
	
	\item Siano $ a,\, b,\, c $ tre interi distinti, e sia $ P $ un polinomio a \emph{coefficienti interi}. Mostrare che le condizioni $$  P(a) = b,\quad P(b) = c,\quad P(c) = a,  $$ non possono essere soddisfatte simultaneamente.

	\item Alberto e Barbara giocano al seguente gioco. Alberto pensa a un polinomio a coefficienti interi non-negativi $ P(x) $. Barbara può fargli $ n $ domande, sotto forma di un intero $ m $. Alberto risponde con $ P(m) $. Qual è il più piccolo $ n $ per cui Barabara ha una strategia vincente?
	
	\item  Trovare il massimo valore possibile di
	\[ P(x) = \frac{(x-14^\pi)(x-12!^e)}{(17-14^\pi)(17-12!^e)} 
	+\frac{(x-12!^e)(x-17)}{(14^\pi-12!^e)(14^\pi-17)} +\frac{(x-17)(x-14^\pi)}{(12!^e-17)(12!^e-14^\pi)} \]
	al variare di $ x $ tra i numeri reali.
	
\end{enumerate}

\newpage
\begin{center}
	\vspace*{0,5 cm}
	{\Huge \textsc{Algebra}} \\
	\vspace{0,5 cm}
	\textsc{Novembre 2019 - Salerno}
	\thispagestyle{empty}
	\vspace{0,7 cm}
\end{center}
\normalsize

\begin{enumerate}
	\item  Il polinomio $ x^3 -26x^2 + 203x - 521 $ ha radici $a,\, b,\, c$ calcola 
	\[ a^2 + b^2 + c^2 \qquad\text{e}\qquad (a+b)(b+c)(c+a) \qquad\text{e}\qquad \frac{1}{a+1}+\frac{1}{b+1}+\frac{1}{c+1}\]
	
	\item  Trovare il massimo valore possibile di
	\[ P(x) = \frac{(x-14^\pi)(x-12!^e)}{(17-14^\pi)(17-12!^e)} 
	+\frac{(x-12!^e)(x-17)}{(14^\pi-12!^e)(14^\pi-17)} +\frac{(x-17)(x-14^\pi)}{(12!^e-17)(12!^e-14^\pi)} \]
	al variare di $ x $ tra i numeri reali.
	
	\item Siano $ a,\, b,\, c $ tre interi distinti, e sia $ P $ un polinomio a \emph{coefficienti interi}. Mostrare che le condizioni $$  P(a) = b,\quad P(b) = c,\quad P(c) = a,  $$ non possono essere soddisfatte simultaneamente.
	
	\item Andrea e Federica giocano al seguente gioco. Andrea pensa a un polinomio a coefficienti interi non-negativi $ P(x) $ e Federica cerca di indovinarlo: può però chiedergli solamente il valore che assume il polinomio in un intero $ m $ a sua scelta. Andrea risponde con $ P(m) $. Quante domande servono a Federica, al minimo, per scoprire il polinomio di Andrea?
	
	
\end{enumerate}

\newpage
\begin{center}
	\vspace*{0,5 cm}
	{\Huge \textsc{Algebra}} \\
	\vspace{0,5 cm}
	\textsc{Mai successo}
	\thispagestyle{empty}
	\vspace{0,7 cm}
\end{center}
\normalsize
\begin{enumerate}
	\item Find the minimum value of
	\[ x^{2019} + \frac{1}{x^{2020}} \]
	for $ x > 0 $.
	
	\item Mostrare che per ogni numero naturale $ n \geq 1 $ vale
	\[ 1 \cdot 3 \cdot 5 \cdots (2n-1) \leq n^n \]
	
	\item Let $ a $, $ b $, $ c $ be the sides of a triangle, and $ T $ its area. Prove: $ a^2+b^2+c^2 \geq 4\sqrt{3}T $. In what case does
	equality hold? (IMO, 1961)
	
	\item Nesbitt.
	\[ \frac{a}{b+c} + \frac{b}{c+a} + \frac{c}{a + b} \geq \frac{3}{2}. \]
	
	\item Trovare il massimo di $ x(x + 3y)(x + 5z) $, al variare di $ x, \, y, \, z $ tra i reali positivi, sotto il vincolo che $ x + y +z = 1 $.
	
\end{enumerate}

\newpage
\begin{center}
	\vspace*{0,5 cm}
	{\Huge \textsc{Algebra}} \\
	\vspace{0,5 cm}
	\textsc{Dicembre 2019 - Pavia}
	\thispagestyle{empty}
	\vspace{0,7 cm}
\end{center}
\normalsize

\begin{enumerate}
	\item  Siano $ a, \, b, \, c $ le radici del polinomio $ x^3 -2019x^2 + 2020x - 42 $, calcola 
	\[ a^2 + b^2 + c^2 \qquad\text{e}\qquad (a+b)(b+c)(c+a)\]
	
	\item  Trovare il massimo valore possibile di
	\[ P(x) = \frac{(x-14^\pi)(x-12!^e)}{(17-14^\pi)(17-12!^e)} 
	+\frac{(x-12!^e)(x-17)}{(14^\pi-12!^e)(14^\pi-17)} +\frac{(x-17)(x-14^\pi)}{(12!^e-17)(12!^e-14^\pi)} \]
	al variare di $ x $ tra i numeri reali.
	
	\item Siano $ a,\, b,\, c $ tre interi distinti, e sia $ P $ un polinomio a \emph{coefficienti interi}. Mostrare che le condizioni $$  P(a) = b,\quad P(b) = c,\quad P(c) = a,  $$ non possono essere soddisfatte simultaneamente.
	
	\item Andrea e Bernardo giocano al seguente gioco. Andrea pensa a un polinomio a coefficienti interi non-negativi $ P(x) $ e Bernardo cerca di indovinarlo: può però chiedergli solamente il valore che assume il polinomio in un intero $ m $ a sua scelta. Andrea risponde con $ P(m) $. Quante domande servono a Bernardo, al minimo, per scoprire il polinomio di Andrea?
	
	\item Trovare il massimo di $ x(x + 3y)(x + 5z) $, al variare di $ x, \, y, \, z $ tra i reali positivi, sotto il vincolo che $ x + y +z = 1 $.
	
	
\end{enumerate}

\end{document}
