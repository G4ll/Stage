\documentclass[a4paper]{article}

\makeatletter
\title{Teoria dei Numeri}\let\Title\@title
\author{Andrea Gallese}\let\Author\@author
\date{\today}\let\Date\@date

\usepackage[english]{babel}
\usepackage[utf8]{inputenc}

\usepackage{mathtools}
\usepackage{amssymb}
\usepackage{amsthm}
%\usepackage{faktor}
\usepackage{wasysym}

\usepackage[margin=1.5cm]{geometry}
\usepackage{fancyhdr}
\usepackage{subfig}
\usepackage{multirow}

\usepackage{lipsum}
\usepackage{titlesec}
\usepackage{multicol}
\usepackage{setspace}
\usepackage{mdframed}

% Comando Intitolante
\newcommand{\Intitola}{\begin{center}
		\vspace*{0,5 cm}
		{\Huge \textsc{\Title}} \\
		\vspace{0,5 cm}
		\textsc{\Author} \hspace{1cm} \textsc{\Date}
		\thispagestyle{empty}
		\vspace{0,7 cm}
\end{center}}

% Formato Teoremi, Dimostrazioni, Definizioni
\newtheorem{theorem}{Teorema}[section]
\theoremstyle{remark}
\newtheorem*{remark}{Osservazione}
\theoremstyle{definition}
\newtheorem*{definition}{Definizione}
\renewcommand\qedsymbol{$\clubsuit$}

% Frontespizio e piè di pagina
\pagestyle{fancy}
\fancyhf{}
\rhead{\textsf{\Author}}
\chead{\textbf{\textsf{\Title}}}
\lhead{\textsf{\today}}

% Per avere le sezioni con le lettere
\renewcommand{\thesection}{\Alph{section}}

%Comandi specifici
\newcommand{\Aut}[1]{\mathrm{Aut}\left( #1 \right)}
\newcommand{\Int}[1]{\mathrm{Int}\left( #1 \right)}
\newcommand{\Orb}[1]{\mathcal{O}rb\left( #1 \right)}
\newcommand{\Stab}[1]{\mathcal{S}tab\left( #1 \right)}

\newcommand{\fun}[5]{\begin{align*}
	#1 \colon #2 &\to #3 \\
	#4 &\mapsto #5
	\end{align*}}

% indentazione
\setlength{\parindent}{0pt}

% multicols
\usepackage{multicol}
\setlength\columnsep{20pt}
\setlength{\columnseprule}{0,5pt}

% Per disegnare diagrammi commuatativi
\usepackage{tikz-cd}

% Serve a decidere il bullet delle liste puntate
\renewcommand\labelitemi{-}

\begin{document}
%\Intitola
%\small

%\begin{multicols}{2}

%
%\begin{center}
%	\textsc{Divisione con Resto}
%\end{center}
%
%\textsf{Cos'è, proprietà.}\\
%
%\textbf{Problema 1.} Dimostrare che $2^{32 }+1 $ è divisibile per $ 641 $.\\
%
%\textsf{La soluzione è assolutamente magica: $ 641 = 2^4 + 5^4 = 2^7 \cdot 5 -1 $, pertanto $ 641 \mid 2^{28}\cdot 5^4 - 1 $ e $ 641 \mid  2^{28}\cdot 5^4 + 2^{32} $.}\\
%
%\textbf{Problema 2.} [Feb 13] Determinare tutte le terne di interi strettamente positivi $ (a, b, c) $ tali che
%\begin{itemize}
%	\item  $ a \leq b \leq c $;
%	\item  $ MCD (a, b, c) = 1 $;
%	\item  $ a $ è divisore di $ b + c $, $ b $ è divisore di $ c + a $ e $ c $ è divisore di $ a + b $.
%\end{itemize}
%  
%  
%\begin{center}
%	\textsc{Numeri primi}
%\end{center}
%\textsf{Definizione}\\
%
%\textbf{Problema 3.} $ 4^{545} + 545^4 $ è primo?
%
%\begin{center}
%	\textsc{Teorema Fondamentale dell'Aritmetica}
%\end{center}
%\textsf{Enunciato (che non si capisce)}
%$$  \boxed{2018 = 2 \cdot 1009}  $$
%\textbf{Problema 4.} Trovare tutte le terne di interi $ (a, b, c) $ tali che $$  a^2 + b^2 = 3c^2  $$ 
%\columnbreak
%
%\begin{center}
%	\textsc{Congruenze}
%\end{center}
%\begin{enumerate}
%	\item  Che ora sarà tra esattamente $ 1000 $ ore?
%	
%	\textsf{Definizione, spiegazione, notazione...}
%	
%	\item What are the remainders when $ 3333 + 4444 $ and $ 3333 \cdot 4444 $ are divided by 5?
%	
%	\textsf{Osservazioni su come si comporta la congruenza contro le operazioni}
%	
%	\item What is the remainder when $ 2015^{2015} $ is divided by 2014?
%	
%	\textsf{Qualche parola sulla divisione}
%	
%	\item Is $ 21^{100} - 12^{100} $ a multiple of 11?
%	\item What is the remainder when $ 7^{2015} $ is divided by 48?
%	\item What are the last two digits of the integer $ 17^{198} $?
%	\item Criteri di congruenza. Per $ 2^n $, $ 3 $, $ 9 $, $ 11 $.
%\end{enumerate}
%
%\end{multicols}
%
%\textbf{Problemi da proporre.}
%\begin{multicols}{2}
%	\begin{enumerate}
%		\item Dimostrare che, se $ p $ e $ p^2 +2 $ sono primi, anche $ p^3 +2 $ lo è.
%		
%		\item $  1 \cdot 3 \cdot 5 \cdots 2013 + 2 \cdot 4 \cdot 6 \cdots 2014  $ è divisibile per 2015?
%		
%		\item Trovare il resto quando:
%		\begin{itemize}
%			\item [(a)] 555 è diviso per 13 \\
%			\item [(b)] $ 555^2$ è diviso per 13\\
%			\item [(c)] $ 156 \cdot 167 $ è diviso per 7 \\
%			\item [(d)] $ 24^{50} - 15^{50} $ è diviso per 13\\
%			\item [(e)] $ 1 + 2 + \dots + 2007 $ è diviso per 1000\\
%			\item [(f)] $ 5^{15} $ è diviso per 128\\
%			\item [(g)]$  7^{7^7} $ è diviso per 10\\
%			\item [(h)] $ 12^9 $ è diviso per 1000\\
%		\end{itemize}
%		
%		
%		\item Quante sono le coppie di interi ordinate $ (x, y) $ tali che $ xy = 4(y^2 + x) $? [Feb 13] \\
%		
%		\item Dato un qualsiasi intero positivo $ n $, chiamiamo ciclostilato di $ n $ il numero che si ottiene concatenando 2012 scritture di $ n $ (in base 10). Per esempio il ciclostilato di 314 è $ 314314314\dots314 $, dove le cifre “314” si ripetono 2012 volte.
%		\begin{itemize}
%			\item [(a)] Determinare tutti gli interi positivi $ m $ tali che il ciclostilato di $ m $ sia multiplo di 9.
%			\item [(b)] Determinare tutti gli interi positivi $ m $ tali che il ciclostilato di $ m $ sia multiplo di 11.
%		\end{itemize}
%	
%		
%		\item $ \dfrac{1}{x} + \dfrac{1}{y} = \dfrac{1}{6} $
%		
%		\item Trovare tutte le coppie di interi tali che $ a^2 + b^2 = 8070 $.
%		
%		\item Quante sono le coppie di interi positivi $(a,b)$ con $a<b$ tali che $MCD(a,b)=2$ e $mcm(a,b)=660$?
%		
%		\item $ (n, n+1) = 1 $
%		
%		\item $ (a, b)[a, b] = ab $
%		
%	\end{enumerate}
%\end{multicols}
%
%\newpage
%\begin{multicols}{2}
%	
%	
%	\begin{center}
%		\textsc{Divisione con Resto}
%	\end{center}
%	
%	\textsf{Possiamo vedere ogni numero attraverso la struttura moltiplicativa imposta dall'altro:}\\
%	
%	\textbf{Teorema 1.}  Per ogni coppia di interi $ a \geq b $ esistono e sono unici interi $ q, r $ (detti quoziente e resto), con $ 0 \leq r < b $, tali che: \[ a = b \cdot q +r \]
%	
%	\textbf{Divisibilità.} Definizione e proprietà: transitiva, somma, coprimi. \\
%	
%	\textbf{Problema 1.} [Feb 13] Determinare tutte le terne di interi strettamente positivi $ (a, b, c) $ tali che
%	\begin{itemize}
%		\item  $ a \leq b \leq c $;
%		\item  $ MCD (a, b, c) = 1 $;
%		\item  $ a $ è divisore di $ b + c $, $ b $ è divisore di $ c + a $ e $ c $ è divisore di $ a + b $.
%	\end{itemize}
%
%	\begin{center}
%		\textsc{Congruenze}
%	\end{center}
%	\begin{enumerate}
%		\item  Che ora sarà tra esattamente $ 1000 $ ore?
%		
%		\textsf{Definizione, spiegazione, notazione...}
%		
%		\item What are the remainders when $ 3333 + 4444 $ and $ 3333 \cdot 4444 $ are divided by 5?
%		
%		\textsf{Osservazioni su come si comporta la congruenza contro le operazioni}
%		
%		\item What is the remainder when $ 2019^{2019} $ is divided by 2018?
%		
%		\textsf{Qualche parola sulla divisione}
%		
%		\item Is $ 21^{100} - 12^{100} $ a multiple of 11?
%		
%		\item What is the remainder when $ 7^{2019} $ is divided by 48?
%
%	\end{enumerate}
%	
%	\textbf{Problema 2.} Trovare la cifra delle unità di $ 3^{3^{\dots^3}} $, dove la cifra $ 3 $ compare ben cento volte. \\
%	
%	\textbf{Problema 3.} Trovare tutti i naturali $ n $ tali che \[ n + d(n) + d(d(n)) = 2019 \]
%	
%	\textsf{Qua si parla di criteri di divisibilità e criteri di congruenza!}\\
%	
%	\textbf{Problema 4.} Trovare tutte le terne di interi $ (a, b, c) $ tali che $$  a^2 + b^2 = 3c^2  $$
%	
%	\begin{center}
%		\textsc{Numeri primi}
%	\end{center}
%	\textsf{Definizione. Albero?}\\
%	
%	\textbf{Problema 5.} 641 è primo? 2019 è primo? $ 1{00\dots 00}1 $ con 2019 zeri è primo? $ 4^{545} + 545^4 $ è primo? $ n^5 + n^4 +1 $ è primo? \\
%	
%	\textsf{Decidere se un numero è primo o meno non è, generalmente, un problema facile. Quando si chiede di farlo in genere si cerca una fattorizzazione esplicita!}\\
%	
%	\textbf{Infinitudine.} \\
%	
%	\textbf{Problema 6.} Quali primi si possono scrivere come somma di due quadrati? \\
%
%	\textbf{Teorema Fondamentale dell'Aritmetica.} \\
%	
%	\textbf{Problema 7.} Trovare tutte le terne di naturali $ (p, m, n) $ con $ p $ primo tali che
%	\[ p^n +144 = m^2 \]
%	
%	
%	\textsf{Non dovrebbe avanzare tempo, ma, se succede, si può parlare di quel poco che si sa sui numeri primi: ci sono buchi arbitrariamente grandi, tra $ n $ e $ 2n $ c'è almeno un numero primo, primi gemelli, primi di SG...}
%	
%	
%	
%\end{multicols}
%
%\newpage
%
%\begin{multicols}{2}
%	\textsf{Voglio una lezione un po' più avanzata e un po meno specifica. L'argomento vorrebbe essere qualcosa del tipo "come si risolvono le diofantee"; l'introduzione è chiara: nessuno ha la più pallida idea di come si risolva una diofantea, in generale, magari con un bell'esempio mortale. Questo lascia intuire che i problemi delle Olimpiadi hanno la magica proprietà di avere sempre una soluzione elementare.}\\
%	
%	\textbf{Esempio mortale.} Sappiamo risolvere, negli interi positivi, 
%	\[ \frac{a}{b+c} + \frac{b}{c+a} + \frac{c}{a + b} = n \]
%	per $ n= 1 $ ma non per $ n=4 $.\\
%
%	\textbf{Teorema.} Siamo invece capaci di risolvere le diofantee lineari
%	\[ ax+by = c. \]
%	
%	\textbf{Problema 2.} [SNS 16] Trovare tutte le coppie di interi $ (x, \, y) $ che risolvono
%	$$  \dfrac{1}{x} + \dfrac{1}{y} = \dfrac{1}{6}.  $$
%	Che cosa succede se metto $ 6^{20} $ al posto di $ 6 $?\\
%	
%	\textsf{Il discorso qui è che la grande strategia di sempre è cercare di fattorizzare l'equazione, perché conosciamo bene la struttura moltiplicativa di $ \mathbb{Z} $. Segue discorso sul numero di divisori.} \\
%	
%	\textbf{Problema 8.} [1991 IMO Short List] Find all positive integer solutions to $$  3^x + 4^y = 5^z.  $$
%	
%	\textsf{La seconda strategia mistica è guardare che cosa succede all'equazione ``mod $ m $''. Se non sanno le congruenze è potenzialmente un disastro.}\\
%	
%	\textbf{Problema 9.} Determinare quanti sono i numeri interi (relativi) $ n $ tali che
%	\[n^2 + 85n + 2019\]
%	\`e un quadrato perfetto.\\
%	
%	\textsf{Bisogna trovare un modo dolce di introdurre l'idea di stringere tra quadrati.} \\
%	
%	\textbf{Problema 13.} [Febbraio 13] Quante sono le coppie di interi ordinate $ (x, y) $ tali che \[ xy = 4(y^2 + x)? \]
%	
%	\textsf{Se siamo arrivati fino a qui e non è ancora finita, abbiamo sbagliato qualcosa. Del tipo: in realtà i bimbi non stanno capendo nulla perché siamo andati troppo in fretta, in realtà i bimbi dovrebbero cominciare a provare a risolvere la Congettura di Goldbach, in realtà sei nel posto sbagliato... Comunque, possiamo parlare delle terne pitagoriche.}\\
%	
%	\textbf{Teorema.} Trovare tutte le terne di interi positivi $ (a, b, c) $ tali per cui
%	\[ a^2 + b^2 = c^2. \]
%	
%	
%	%% --------------------------------------------------------------
%	
%	
%	
%	
%	
%\end{multicols}
%
%\textbf{Problemi}.
%\begin{enumerate}
%	\item Find $ 3x^2y^2 $
%	if $ x $ and $ y $ are integers such that $ y^2+ 3x^2y^2= 30x^2 + 517. $ (AIME, 1987)
%	
%	\item The harmonic mean of two positive numbers is the reciprocal of the arithmetic mean of their reciprocals.
%	For how many ordered pairs of positive integers $ (x,\, y) $ with $ x < y $ is the harmonic mean of $ x $ and $ y $
%	equal to $ 6^{20} $? (AIME, 1996)
%	
%	\item The integer $ N $ is positive. There are exactly 2005 ordered pairs $ (x,\, y) $ of positive integers satisfying
%	\[ \frac{1}{x}+ \frac{1}{y} = \frac{1}{N}. \]
%	Prove that $ N $ is a perfect square. (British Mathematical Olympiad, 2005)
%	
%	\item [GHF] Non ci sono soluzioni intere di $ a^2b + ab^2 = x^3 $.
%	
%	\item Determinare per quanti valori interi di a il numero $ 2a^2+27a+91 $ risulta un quadrato perfetto.
%	
%	\item Determinare tutti gli interi positivi a con questa propriet\`a: comunque si scelga un intero
%	positivo $ n $, il numero $ n(a + n) $ non \`e un quadrato perfetto.
%	
%	\item Determinare quante sono le coppie (p, m) in cui p `e un numero primo (e quindi positivo), m `e
%	un numero intero (relativo) e
%	\[p(m - 78p) = (m - 3)3 + 3p.\]
%	
%	\item  Determinare tutte le terne $ (p, x, y) $ in cui $ p $ \`e un numero primo e $ (x, y) $ \`e una coppia di numeri
%	interi tali che
%	\[	x^3
%	(x^3 + y) = py^2
%	.\]
%	
%	\item Determinare tutte le coppie $ (p, n) $ in cui $ p $ \`e un numero primo, $ n $ \`e un intero positivo e
%	\[p^8 - p^4 = n^5 - n.\]
%	
%	\item [Cese 15] Determinare tutte le coppie di numeri interi $ (a, b) $ che risolvono l’equazione \[a^3 + b^3 + 3ab = 1.\]
%	
%	\item [Senior 19] Determinare tutte le coppie $ (a,\, b) $ di interi positivi per cui esiste un numero primo $ p $ tale che
%	\[9^a + 3^a - 2 = 2p^b.\]
%	
%	\item  Trovare tutte le terne di numeri interi $ (a, b, c) $ tali che
%	\[ a^2 + b^2 = 3c^2. \]
%	
%	% discesa infinita
%	
%	\item  [Ungheria] Let $ x $, $ y $, and $ z $ be rational numbers satisfying
%	\[x^3 + 3y^3 + 9z^3 - 9xyz = 0.\]
%	
%	% stringere fra quadrati
%	% vorrei anche un esercizio in cui questa cosa succede in modo più furbo: devo risolvere una quadratica e per farlo mi serve stringere fra quadrati
%	
%	
%	\item  Determinare quante sono le coppie $ (m, n) $ di numeri interi (relativi) tali che
%	\[n^2 - 6n = m^2 + m - 12.\]
%	
%	% fattorizzare: JFFT, polinomio da fattorizzare
%	
%	\item  Trovare tutte le terne di naturali $ (p, m, n) $ con $ p $ primo tali che
%	\[ p^n +144 = m^2. \]
%	
%	
%	% disuguaglianze per stringere i casi
%	
%	\item  Find all positive integers $ n $ such that $$  1 + 2^2 + 3^3 + 4^n  $$ is a perfect square.\\
%	
%	\item  Trovare tutte le terne di interi $ (x, y, z) $ per cui vale, per ogni $ n $, che \[ x^n + y^n + z^n = 0. \]
%	
%	\item  \[ m(n+3) = n^4 + 2016 \]
%	
%	
%	
%	% equazioni che bisogna saper risolvere: lineari, pitagorica, pell (che però non ti serviranno mai a nulla)
%\end{enumerate}
%
%\newgeometry{left=3cm,bottom=0.1cm}
%\Intitola
%\normalsize
%
%
%
%\begin{enumerate}
%	\item Dimostrare che, se $ p $ e $ p^2 +2 $ sono primi, anche $ p^3 +2 $ lo è.
%	
%	\item $  1 \cdot 3 \cdot 5 \cdots 2013 + 2 \cdot 4 \cdot 6 \cdots 2014  $ è divisibile per 2015?
%	
%	\item Trovare il resto quando:
%	\begin{itemize}
%		\item [(a)] 555 è diviso per 13 \\
%		\item [(b)] $ 555^2$ è diviso per 13\\
%		\item [(c)] $ 156 \cdot 167 $ è diviso per 7 \\
%		\item [(d)] $ 24^{50} - 15^{50} $ è diviso per 13\\
%		\item [(e)] $ 1 + 2 + \dots + 2007 $ è diviso per 1000\\
%		\item [(f)] $ 5^{15} $ è diviso per 128\\
%		\item [(g)]$  7^{7^7} $ è diviso per 10\\
%		\item [(h)] $ 12^9 $ è diviso per 1000\\
%	\end{itemize}
%	
%	
%	\item Quante sono le coppie di interi ordinate $ (x, y) $ tali che $ xy = 4(y^2 + x) $? [Feb 13] \\
%	
%	\item Dato un qualsiasi intero positivo $ n $, chiamiamo ciclostilato di $ n $ il numero che si ottiene concatenando 2012 scritture di $ n $ (in base 10). Per esempio il ciclostilato di 314 è $ 314314314\dots314 $, dove le cifre “314” si ripetono 2012 volte.
%	\begin{itemize}
%		\item [(a)] Determinare tutti gli interi positivi $ m $ tali che il ciclostilato di $ m $ sia multiplo di 9.
%		\item [(b)] Determinare tutti gli interi positivi $ m $ tali che il ciclostilato di $ m $ sia multiplo di 11.
%	\end{itemize}
%	
%	
%	\item Trovare tutte le coppie ordinate $ (x, y) $ di interi tali che $$  \dfrac{1}{x} + \dfrac{1}{y} = \dfrac{1}{6}  $$
%	
%\end{enumerate}
%
%\newgeometry{left=3cm,bottom=0.1cm}
%\begin{center}
%	\vspace*{0,5 cm}
%	{\Huge \textsc{Teoria dei Numeri}} \\
%	\vspace{0,5 cm}
%	\textsc{\Author} \hspace{1cm} \textsc{25 luglio 2018 - Palermo}
%	\thispagestyle{empty}
%	\vspace{0,7 cm}
%\end{center}
%\normalsize
%
%\begin{enumerate}
%	\item Dimostrare che, se $ p $ e $ p^2 +2 $ sono primi, anche $ p^3 +2 $ è primo.\\
%	
%	\item $  1 \cdot 3 \cdot 5 \cdots 3017 + 2 \cdot 4 \cdot 6 \cdots 3018  $ è divisibile per 3019?\\
%	
%	\item Esistono due interi positivi $ x $ e $ y $ tali che $ x+y $, $ x+2y $ e $ y+2x $ sono tutti e tre quadrati perfetti?\\
%	
%	\item Sia $ A = 77\dots 77 $ l'intero formato da 7777 cifre uguali a 7. Prendiamo la somma delle cifre di $ A^A $, dopodiché prendiamo la somma delle cifre di quanto ottenuto e ripetiamo il procedimento fino a quando non ci rimane un numero di una sola cifra. Di che numero si tratta?\\
%	
%	\item Un intero positivo $ n $ si dice \textit{bilanciato} se è prodotto di un numero pari di fattori primi. Dati due interi positivi $ a$ e $ b $, consideriamo il polinomio $ P(x) = (x+a)(x+b) $.
%	Dimostrare che:
%	\begin{itemize}
%		\item [(i)] Si possono scegliere $ a $ e $ b $ in modo che $ P(1),\, P(2),\,\dots ,\, P(2018) $ siano tutti bilanciati.
%		\item [(ii)] Se $ P(n) $ è bilanciato per ogni $ n $, allora $ a = b $.\\
%	\end{itemize}
%	
%	\item \emph{Difficile}: \[2^{2^5} +1 {\text{ è primo?}}\]
%	
%\end{enumerate}
%
%\newgeometry{left=3cm,bottom=0.1cm}
%\begin{center}
%	\vspace*{0,5 cm}
%	{\Huge \textsc{Teoria dei Numeri}} \\
%	\vspace{0,5 cm}
%	\textsc{Terni}
%	\thispagestyle{empty}
%	\vspace{0,7 cm}
%\end{center}
%\normalsize
%
%
%
%
%\begin{enumerate}
%	\item Trovare il resto quando:
%	\begin{itemize}
%		\item [(a)] 555 è diviso per 13
%		\item [(b)] $ 555^2$ è diviso per 13
%		\item [(c)] $ 156 \cdot 167 $ è diviso per 7
%		\item [(d)] $ 24^{50} - 15^{50} $ è diviso per 13
%		\item [(e)] $ 1 + 2 + \dots + 2007 $ è diviso per 1000
%		\item [(f)] $ 5^{15} $ è diviso per 128
%		\item [(g)]$  7^{7^7} $ è diviso per 10
%		\item [(h)] $ 12^9 $ è diviso per 1000 \\
%	\end{itemize}
%	
%	\item Esistono due interi positivi $ x $ e $ y $ tali che $ x+y $, $ x+2y $ e $ y+2x $ sono tutti e tre quadrati perfetti? \\
%	
%	\item (Febbraio 2012.) Dato un qualsiasi intero positivo $ n $, chiamiamo ciclostilato di $ n $ il numero che si ottiene concatenando 2012 scritture di $ n $ (in base 10). Per esempio il ciclostilato di 314 è $ 314314314\dots314 $, dove le cifre “314” si ripetono 2012 volte.
%	\begin{itemize}
%		\item [(a)] Determinare tutti gli interi positivi $ m $ tali che il ciclostilato di $ m $ sia multiplo di 9.
%		\item [(b)] Determinare tutti gli interi positivi $ m $ tali che il ciclostilato di $ m $ sia multiplo di 11. \\
%	\end{itemize}
%	
%	\item Se $ 2n + 1 $ e $ 3n + 1 $ sono quadrati perfetti, allora $ 5n + 3 $ non è un primo. \\
%	
%	\item[$ \star. $] (GAS 2014.) Ellisseo e i suoi compagni giungono nella grotta di Polinomio e scoprono ben presto che il ciclope possiede una
%	gran quantità di pecore. Dividendo le pecore in gruppi da 5, ne avanzano 3, mentre dividendole in gruppi da 7 ne
%	rimangono 2; infine dividendo le pecore in gruppi da 11, ne avanzano 7. Sapendo che ogni pecora ha 11 agnelli e che
%	tutti gli agnelli sono meno di 5000, quanti sono gli agnelli?
%	
%\end{enumerate}

\newgeometry{left=3cm,bottom=0.1cm}
\begin{center}
	\vspace*{0,5 cm}
	{\Huge \textsc{Teoria dei Numeri}} \\
	\vspace{0,5 cm}
	\textsc{Salerno}
	\thispagestyle{empty}
	\vspace{0,7 cm}
\end{center}
\normalsize




\begin{enumerate}
	\item Trovare tutte le coppie $ (x, y) $ di interi relativi per cui vale
	\[ 24x + 35y = 17. \]
	
	\item Trova $ 3x^2y^2 $,
	sapendo che $ x $ e $ y $ sono interi tali che $ y^2+ 3x^2y^2= 30x^2 + 517. $ (AIME, 1987)\\
	
	\item Sia $ N $ un intero positivo. Sappiamo che ci sono esattamente 2005 coppie ordinate $ (x,\, y) $ di interi positivi che soddisfano
	\[ \frac{1}{x}+ \frac{1}{y} = \frac{1}{N}. \]
	Dimostrare che $ N $ è un quadrato perfetto. (British Mathematical Olympiad, 2005)\\
	
	\item  Determinare quante sono le coppie $ (m, n) $ di numeri interi (relativi) tali che
	\[n^2 - 6n = m^2 + m - 12.\]
	
	\item  Trovare tutte le terne di numeri interi $ (a, b, c) $ tali che
	\[ a^2 + b^2 = 3c^2. \]
	
\end{enumerate}

\end{document}
